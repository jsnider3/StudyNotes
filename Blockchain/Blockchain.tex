\documentclass{article}
\usepackage{graphicx}
\usepackage{fullpage}

\begin{document}

\title{Blockchain Notes}
\author{Josh Snider}

\maketitle

%%\begin{abstract}
%The abstract text goes here.
%\end{abstract}

\section{Introduction}
These are my notes from reading the \textit{Blockchain Applications A Hands-On Approach} book by Arshdeep Bahga and Vijay Madisetti. A major goal of reading this book, is to write a blog post based on it, either as a review of the book or as a summary of topics it discusses.

\subsection{Book Contents}
The book starts with an overview of the blockchain and some application templates that can be done with it. It then goes to talk about Ethereum, including how to set up a development environment for it. It then goes to talk about services that can be built on the blockchain such as smart contracts, decentralized applications, mining, whisper, and swarm, before wrapping up with an overview of some more advanced topics involving the blockchain. The book requires minimal understanding of the blockchain on the part of the reader, but assumes a competency with programming. There's a companion website at blockchain-book.com that has downloadable copies of the source code used in the book.

\subsection{About the Authors}

Both of the authors are academics, Vijay's a professor and Arshdeep's a researcher. This book is actually part of a \textit{Hands-On Approach} series which has already covered cloud computing and big data.

\section{Blockchain Basics}

%1.1
The main idea behind the blockchain is to create a platform for handling transactions that doesn't depend on a central hub like banks. The first cryptocurrency using a blockchain was Bitcoin which was introduced in 2008 by Satoshi Nakamoto. Without a central authority, there needs to be a new way to prevent people from spending the same money more than once, the Bitcoin solution requires two things, every transaction being announced publically and a system for agreeing on the order of transactions. This ledger works as a chain of blocks or "blockchain" if you will. The process by which blocks are added is called "mining", it involves a node in the network collected the new transactions and then stamping its block with a proof-of-work. In order to incentivize "mining", blockchains routinely award money to the node who mines the block that gets added to the chain.

%1.1.1
In 2013, Vitalik Buterin proposed a new type of blockchain network called Ethereum. The main idea is to have a single-programmable blockchain that could be shared by multiple applications as opposed to different cryptocurrencies having different blockchains. Each application takes the form of a "smart contract". Gavin Wood was the first to create a functional implementation of Ethereum in 2014 which included the programming language Solidity.

%1.1.2
Each block has four fields, a timestamp, the hash of the previous block, the hash of the block itself (called a \textit{nonce} value), and the hash of the root of the Merkle tree containing the transactions. This means the the block technically does not contain the transactions it has.

%1.1.3
So, what are the core features of the block chain?
\begin{enumerate}
\item \textbf{Immutability} - Once a transaction is recorded in a block, it can't be deleted or altered unless a majority of nodes are conspiring to do so.
\item \textbf{No Central Authority} - This reduces the need for transaction fees and speeds up settlement times. Also, makes it impossible to regulate what people spend money on.
\item \textbf{Secure and Transparent} - Transactions are final unless a majority of miners collude to change them. The entire process by which blockchains run are largely specified in publically accessible white papers and implemented by open-source programs run by the public.
\item \textbf{Privacy} - While all of the transactions are recorded publically, the private key for each party is kept secret and there's no link between the public key and the identity of the transactors.
\item \textbf{Scalable and Available} - The blockchain is replicated amongst nodes to ensure, that even as they leave and join the network the blockchain is preserved.
\end{enumerate}

\section{Example Smart Contract}

%1.2
Let's consider an escrow contract between two people. The buyer deposits money in an escrow account, the seller then gives the buyer the item, and the escrow account releases the payment to the seller. This can be done in Solidity as a smart contract quite easily. I'm not going to rehash the example completely, but the key points are that contract accounts are different from user accounts, that distributed apps can be used to make nice interfaces to contracts, and that interactions with the contract are validated by miners on the blockchain.

\section{A Blockchain Stack}

%1.3
Blockchain is good at maintaining all the transactions in a network, but it's not particularly good at storing large volumes of data or peer-to-peer messaging. That is Ethereum works as a decentralized computing network, but we need something else for decentralized storage, and decentralized messaging. The book recommends Swarm and Whisper respectively.

\subsection{Decentralized Computing - Ethereum}
%1.3.1

Ethereum uses a virtual machine for its execution environment. This environment is called the Ethereum Virtual Machine (EVM), having each node in the network run the same calculation is not efficient, but is necessary to maintain consensus. Ethereum has two forms of accounts, one called Externally Owned Accounts for normal users and contract accounts whose behavior is controlled by associated contract code. You might be worried that letting people compute on the blockchain would bog it down with endless calculations, fortunately there is a solution. Senders are charged a gas fee which increases the more effort their transaction takes. Running on a virtual machine means that the programming language used can be anything as long as it can be compiled to the EVM bytecode.

\subsection{Decentralized Storage - Swarm}
%1.3.2
Decentralized storage and content distribution network, works as a redundant store of Dapp code. Being peer-to-peer means it lacks a single point-of-failure and makes it resistant to DDoS

\subsection{Decentralized Messaging - Whisper}
%1.3.3
Transient messages between Dapps, which works by subscribing to topics.

\section{Blockchain Applications}

%1.5
Everyone loves technology, but only useful technology becomes as popular as blockchain. What are some of the fields that can use blockchain?
\begin{itemize}
\item FinTech
\item IoT
\item Industrial and Manufacturing
\item Assets and Inventory
\item Energy
\item Supply Chain
\item Records and Identity
\item Healthcare
\end{itemize}

A lot of the uses talked about in the book seem like hype to me, either where a centralized-approach would work due to the ready available of a central organization, a need for robustness not strong enough to require a cryptographically-secure ledger, or being better suited to a more mundane Internet of Things approach. This is a weakness of the book that my blog post should address, but I'll try to find some examples of blockchain in action where it seems
like a good fit. Although if we assume that everyone involved is already an Ethereum user(?), the amount of applications where a smart contract based app is appropriate increases massively.

\section{Blockchain Application Templates}
%2
Previous chapter introduced the main components of a blockchain stack, a decentralized computing platform (Ethereum), a decentralized messaging platform (Whisper), and a decentralized storage platform (Swarm), with smart contracts and decentralized apps (Dapps) rounding out the stack.

Dapps run on an individual Ethereum node, while Whisper and Swarm work with clients on each node, but communicate outside of the blockchain network.

The design process for a blockchain app is to list the people or concepts involved and analyze their relationships. This is used to create a smart contract,
then if it is necessary for usability we create a Dapp using HTML and Javascript.

\subsection{Analyzing a Crowdfunding Campaign}
%2.2
What do we want from a crowdfunding campaign app? We want a campaigner to be able to accept donations from anyone and then if the amount of donations meets a target by a deadline, transfer the donations to the campaigner's EOA. This works well as a smart contract, the smart contract is initialized by the campaign organizer with a deadline and goal, backers fund the campaign, and at the end the organizer checks if the funds have been reached, which either triggers a refund or gives the organizer the money. The book has an implementation of this in chapter six.

This is actually a good non-hype example of a blockchain app, mainly because the alternative is Kickstarter which charges an annoying amount of fees.

\subsection{Application Templates}
%2.3
\begin{itemize}
\item Many-to-one - Like the crowdfunding example, there's a contract owner and every user interacts with them in the same way, for example ticket sellers, crowdfunders, etc.
\item Many-to-one - similar to the above but the variables in the smart contract are used to control an IoT device.
\item Many-to-many - The standard peer-to-peer system. Can be used for buyers/sellers, uploaders/downloaders, etc.
\item One-to-one - Financial transactions, they can be set up to automatically transact with the users' accounts.
\end{itemize}


\end{document}