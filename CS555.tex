\documentclass{scrartcl}
\usepackage{enumitem}
\begin{document}
\title{CS 555 Final Exam Study Notes}
\author{Josh Snider}
\date{2015/05/11}
\maketitle
\section*{Slide 7 Notes}
\begin{itemize}
\item What are the seven layers? (This was a midterm question, possibly also on
 the final)
\begin{itemize}
\item Application
\item Presentation
\item Session
\item Transport
\item Network
\item DLC
\item Physical
\end{itemize}
\item Types of switching
\begin{enumerate}
\item Message switching - entire message is sent as one giant block, sent from 
host to host to host. Kind of stupid.
\item Packet switching - each packet in the message is sent individually, 
real-time, reliable, prioritization
\item Circuit switching - dedicated circuit is created between beginning and 
end, phones.
\item Leased lines - permanently dedicated circuit, heavy constant traffic. 
See Spread Networks.
\end{enumerate}
\item Network functions
\begin{itemize}
\item Addressing - mapping names to addresses. translation done by nameservers.
\item Routing - selecting path for messages to take.
\item Congestion control
\end{itemize}
\item Types of Routing
\begin{enumerate}
\item Connection-oriented - establish a virtual circuit and send all data over 
it. Guarantees packets arrive in-order.
\item Connectionless - Like IP, each packet is routed seperately as datagrams.
 Has some issues that need to be handled.
\item Source routing - done as homework, not practical because it assumes 
everyone knows everything
\end{enumerate}
\item Ways to control congestion
\begin{enumerate}
\item Tell hosts to shut up.
\item Allocate more buffers.
\item Drop packets (connectionless only)
\item Require use of sliding window (connection-oriented only)
\item Reroute packets around congestion as if they were cars in traffic 
(connectionless only)
\end{enumerate}
\end{itemize}
\subsection*{Internetworking}
\begin{itemize}
\item Definition: routing among networks as opposed to within them, can be 
either connection-oriented or connectionless
\item Open Systems Interconnect - An open standard for connection-oriented 
internetworking. Expensive, supplanted by TCP/IP. Relies on X.25.
\item What is X.25?
\begin{itemize}
\item connection-oriented
\item requires X.21 bis and LAPB as layer 1/2 standards
\item Either over switched virtual circuit (SVC) or permanent virtual circuit 
(PVC).
\item Specifies interface between host and network switch. Interfaces between 
switches are left up to carrier.
\item X.25 Virtual Call: DTE sends CALL REQUEST packet, Receiver either replies
 with CALL ACCEPTED or CALL REJECTED. If accepted, communication begins. 
 There's a sliding window which is usually 2 packets wide with packets of 256 
 bytes.
\item X.25 Fast Select: For short transmissions, can send 128 bytes with CALL
 ACCEPT/REJECT.
\item Connection Oriented Networking with X.75. X.75 defines interface between
 two X.25 networks. Makes bigger virtual network.
\end{itemize}
\item What is Asynchronous Transfer Mode?
\begin{itemize}
\item Connection-oriented packet-switched network
\item Sends cells of fixed 53 bytes, 5 bytes header, 48 bytes data.
\item Underlying technology for "Broadband Integrated Services Digital Network",
 which has its own stupid reference model.
\item Has associated hardware AND software. (sounds expensive!)
\item Viewpoints: Integrated access for users, network infrastructure for
 computers, backbone for lesser networks.
\item Reasons for small cell size
\begin{enumerate}
\item Reduced queueing delay
\item Minimize head-of-line blocking
\item Error correction for small cells and headers
\item Minimize jitter
\item Fixed format switching inefficiencies
\end{enumerate}
\item Routing is connection-oriented.
\item Basic element of routing is virtual channel. These are grouped into 
virtual paths.
\item Like X.25 we have PVCs and SVCs.
\item Connection setup is done with SETUP, CALL\_PROCEEDING, CONNECT, 
CONNECT\_ACK, RELEASE, and RELEASE\_COMPLETE messages.
\end{itemize}
\item SONET (Review)
\item Definition: Synchronous Optical NETwork, OC-1 is 51.84 Mbit/s, faster 
signals are made by multiplexing links.
\item Proprietary network protocols: Largely irrelevant, but 
System Network Architecture (SNA) was a popular IBM one and 
Internetwork Packet Exchange (IPX) was a Novell one.
\end{itemize}
\subsection*{The Routing Problem}
\begin{itemize}
\item Can either be done with shortest-path or "optimal" routing, 
based on whether you weight the links.
\item Primitive routing techniques include:
\begin{enumerate}
\item Source routing: Message contains list of nodes that must be visited
 on path to dest.
\item Static routing: predetermined paths that do not change.
\item Flooding: Spam everyone, then everyone spams everyone else. Horribly 
inefficient.
\end{enumerate}
\item Adaptive routing is the good alternative to static routing. Based on
 measuring network in action.
\item Shortest path spanning tree routing, either using Bellman-Ford or 
Dijkstra to compute the shortest path to each other node.
\end{itemize}
\section*{Slides 8 Notes}
\begin{itemize}
\item Metcalfe's Law: Value of a network to a user is proportional to $n^2$,
 where n is number of users.
\item Others argue that it's actually $n\log(n)$, these people have better
 evidence.
\item DARPA created ARPANET in 1968, which introduced RFC's.
\item In the 70's TCP/IP was developed as the protocol suite for ARPANET.
\item IETF now supports dozens of open protocols that make up the Internet
 Protocol Suite (IPS).
\item TCP/IP inventors were Vinton Cerf and Robert Kahn.
\item internets are made out of subnetworks.
\item A subnet contains bridges to relay frames within and routers to forward
 packets to other subnets.
\item MTU - Maximum Transfer Unit: Absolute maximum set by protocol, subnets
 can always make it smaller.
\end{itemize}
\subsection*{Internet Protocol (IP)}
\begin{itemize}
\item Connectionless, uses datagrams.
\item Delivers packets on a "best effort" basis with no guarantee of order.
\item Grown in popularity as a directly run protocol.
\item IPv4 uses 32-bit addresses. There were originally 4 classes, but that's
 historic now.
\item Uses a hierarchical domain name system.
\item Classless InterDomain Routing (CIDR) is a modern way of making blocks
 for the internet. I'm not entirely sure what it actually does?
\item Prefix notation, write blocks as 123.456/16, or 123.456.789/24.
\item IP Header contains a TIME TO LIVE (TTL) field which limits how many 
 routers it can go through.
\item Address Resolution Protocol (ARP) - convert network address -> physical
 address.
\item Dynamic Host Configuration Protocol (DHCP) - runtime assignment of IP
 addresses.
\item IP Control Message Protocol (ICMP) - Passes control data between routers.
 Sends error statistics, pings, congestion control, klok times, routings loops
 etc.
\item ping and traceroute are ICMP utilities.
\item Fragmentation - algorithm for breaking up packets in cas MTU is smaller.
\begin{enumerate}
\item Divide data based on MTU.
\item Copy original IP header to IP header of each fragment.
\item Mark fragments specially.
\item Reassembled at destination using header flags.
\end{enumerate}
\item IPv6 (You should use it, IPv4 users are short-sighted fools)
\item Uses MTU discovery instead of IPv4 fragmentation
\item Has mandatory security (explain)
\item Slow adoption, even though IPv4 has run out of room.
\item INTERNET OF ALL TEH THINGZ!
\item IOT = embedded computers connected to internet (GENIUS!)
\item IPv4 doesn't have enough addresses, security is concerning.
\item Software defined networks is a research topic.
\item Gateway: Router that serves a subnet.
\end{itemize}
\subsection*{Security}
\begin{itemize}
\item Layer 2 link encryption, nodes using shared keys sending ciphertext 
between them.
\item Layer 3 end-to-end encryption, body encrypted header not. Communities can share keys apparently?
\item IP Security Protocol (IPSEC). Layer 3 protocol.
\item IPSEC has three primary mechanisms
\begin{enumerate}
\item IP Authentication Header for authentication/integrity.
\item IP Encapsulating Security Payload (ESP) for confidentially using symmetric
key encryption.
\item Manual or automated key distribution - Internet key management protocols
(IKMP) being developed.
\end{enumerate}
\item AH and ESP have two modes:
\begin{enumerate}
\item Transport mode: host to host
\item Tunneling mode: firewall to firewall
\end{enumerate}
\item Firewall is another Layer 3 thing. It either filters things based on application logic or based on packet headers.
\item Routing v forwarding: Routing determines where to send things. Forwarding is the process of sending packets to where routing says it should go.
\item Routing either done by shortest-path (in hops) or collecting data and 
trying to find an optimal solution.
\item Routing Information Protocol (RIP) uses a distance vector and UDP,
Unix builtin, messages exchange routing tables.
\item Open Shortest Path First (OSPF) use link state and IP, load balancing,
router exchange info with Link State Advertisement (LSA) messages.
\item OSPF design considerations
\begin{enumerate}
\item "Open"
\item Must support a variety of distance metrics.
\item Needs to respond dynamically to changes.
\item Should allow TOS (def?) routing
\item Should do load balancing
\item Security of routing updates
\end{enumerate}
\item OSPF supports three kinds of topologies (word choice?)
\begin{enumerate}
\item Point-to-point between two routers.
\item Multi-access network with broadcasting.
\item Multi-access networks without broadcasting.
\end{enumerate}
\item OSPF Protocols
\begin{enumerate}
\item Hello Protocol: Sends heartbeats.
\item Exchange Protocol: To synchronize routing databases.
\item Flooding Protocol: Distribute updates and ACK them.
\end{enumerate}
\item Border gateway Protocol
\begin{enumerate}
\item Internet gateways exchange routing info among admin domains.
\item gateway "advertises" that it can reach certain IP networks and its 
distance to them.
\item Distance metrics not standardized
\item Exterior routing through autonomous systems (see below)
\item Uses TCP
\end{enumerate}
\item Autonomous systems: A region of the Internet under the control of a
single entity.
\end{itemize}
\section*{Slides 9 Notes}
\subsection*{Queueing theory}
\begin{itemize}
\item Definition: the mathematical foundation for
understanding/predicting the behavior of packet-switched networks.
\item One Model is a Single Server M/M/1 $\lambda$ is mean arrival rate and
$\mu$ is the mean service rate where $\lambda < \mu$.
\item Next slide confuses me. (Look at again.)
\item Exponential Distribution: PDF $\lambda e^{-\lambda t}$. CDF 
$e^{-\lambda t}$
\item Formulas
\begin{itemize}
\item Utilization factor $\rho = \lambda/\mu$
\item Number in system $N = \rho / (1-\rho)$
\item Time in system (Little's formula) $T=N/\lambda$
\item Waiting time in queue $W=T-1/\lambda$
\item Number in queue $N_q = \lambda W$
\end{itemize}
\item Pollaczek-Khinchin Formula Total time in queueing node $T = \bar{x} +
\frac{\lambda * \bar{x}^2}{2(1-\rho)}$, where $\bar{x}$ = average service time,
$\lambda$ = average arrival rate, $\rho$ = utilization, and $x^2$ = second
moment of service (?).
\item WORK SOME EXAMPLES YOU LAZY FUCK!
\end{itemize}
\subsection*{Transport Protocols}
\begin{itemize}
\item Transport is the fourth layer of the seven layer model. It handles
end-to-end error and flow control, sometimes congestion as well.
\item Transport layers must provide reliable/consistent service in the face of:
\begin{enumerate}
\item Connectionless Networking
\item Circuit failures
\item Packet reordering
\end{enumerate}
\item Elements of transport protocols
\begin{enumerate}
\item Connection Management
\item Segment and reassemble application data
\item Recover from network failures
\item Error control and flow control
\item Multiplex multiple transport connections to one network connections.
\item Splitting and recombining: map one transport connection to multiple
network connections.
\end{enumerate}
\item There are two main internet transport protocols, User Datagram Protocol
(UDP) and Transmission Control Protocol (TCP). UDP is unreliable, TCP is the
opposite.
\end{itemize}
\subsubsection*{Stuff about TCP}
\begin{itemize}
\item Provides connection for connectionless IP.
\item Reliable and ordered.
\item Flow control via sliding windows
\item Uses Ports: In Berkeley socket = IP address + port.
\item Full-duplex transmission (Sends ACKs with responses)
\item Unique ID = (Sending IP/port, Receiving IP/port)
\item Delayed Duplicate Sequence Numbers: Very old packets from previous
connections may show up as if to appear in sequence for new connections.
How to prevent this? Either prevent sequence number reuse or purge old
packets from the network.
\item TCP uses a three-way handshake to setup connections and to release
connections.
\item TCP Startup, meet on pre-agreed port and then usually negotiate a 
switch to a different one.
\item Uses slow-start to avoid congestion, start with minimum window size
of 1 MTU/64KB and every round trip we double the window size (unless
we see congestion). When we reach a threshold, we then start increasing
one MTU at a time instead of doubling. If we experience congestion, cut 
the threshold in half and begin slow-start again.
\item TCP keeps a couple timers around, one for retransmission, one for
persistence, one for keep alive, and one for TIMED WAIT.
\item Retransmission timer needs to be dynamically adjusted due to 
network differences. To do this we maintain an RTT variable to estimate the
round trip time. Each time we receive a non-timed out segment in time M,
we set $RTT \leftarrow \alpha RTT + (1+\alpha)M$, where $\alpha < 1$. 
We also track a deviation estimator $D$ and update it as $D \leftarrow 
\alpha D + (1 - \alpha)|RTT - M|$. We set the timeout to RTT + 4D.
\item What if we retransmit a segment and then get an ACK for it? Using
Karn's algorithm we don't update the RTT, but we double the timeout.
\item What if we need to send data without a full segment? Use the "push"
feature. This might be inefficient, so Nagle's algorithm accumulates data
in a buffer while waiting for an ACK and then sends whatever it has when
we get ACK'd.
\item TCP kinda sucks for high-performance networks due to knee-jerk
congestion control which cuts performance in half whenever it loses
something.
\item Network Address Translation: Use an IPv4 address to "front" for
many hosts and provide them private IPs from reserved space.
Then use TCP port multiplexing to identify who the thing is actually
intended for. Doesn't work for peer-to-peer. Since these hosts don't 
have public IPs, it's slightly more secure since outsiders can't just
randomly call them up and give them some viruses.
\item TODO Work Examples
\end{itemize}
\end{document}
